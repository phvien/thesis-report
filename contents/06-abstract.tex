% \begin{Huge}
% {\raggedleft \textbf{Abstract}}
% \end{Huge}\vspace{1.6cm}
\chapter*{\Huge Abstract}
In this thesis, we present the application of Support Vector Machine (SVM) in the classification of disease on orange leaf. Our system extracts the Scale-Invariant Feature Transform (SIFT) features presented in the Bag-of-Visual-Words (BoVW) model, the Color features, the Histogram of Oriented Gradient (HOG) features, GIST and Residual Networks features from images of diseased orange leaves and learning the SVM model from features of SIFT-BoVW, Color, HOG, GIST and ResNet. The numerical test results on the real dataset with 511 images collected at the Hau Giang and Dong Thap provinces show that our system achieves an accuracy 90.84\%, more accurate than the K-Nearest Neighbors (KNN) model when using the same features, but only acchieves accurate an of 28.16\%. \\

\noindent \textbf{Keywords}--\emph{Image  classification, classification of disease on orange leaf, SIFT features, BoVW model, Color features, HOG features, GIST features, ResNet features, Support Vector Machines (SVM) model.}