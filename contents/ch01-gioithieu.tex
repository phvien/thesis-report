\chapter{Giới thiệu} \label{chapter01}

\section{Đặt vấn đề}\label{datvande}
Nhắc đến cái tên Đồng bằng sông Cửu Long (ĐBSCL), không thể không nhắc đến sông Mê Kông, một trong những con sông lớn nhất trên Thế giới bắt nguồn từ cao nguyên Thanh Tạng chảy qua Trung Quốc, Lào, Myanma, Thái Lan, Campuchia và chảy vào Việt Nam chia thành hai sông là sông Tiền (Tiền Giang) và sông Hậu (Hậu Giang) rồi tỏa ra thành chín nhánh đổ ra biển (Biển Đông) thành chín cửa: Tiểu, Đại, Ba Lai, Hàm Luông, Cổ Chiên, Cung Hầu, Định An, Ba Thắc và Trần Đề. Chín nhánh này như chín con rồng uốn lượn nên ở Việt Nam sông Mê Kông được đặt tên là sông Cửu Long. Qua nhiều năm, hai cửa sông Ba Thắc và Trần Đề bị bùn đất bồi lắp và biến mất, do đó sông Cửu Long giờ chỉ còn bảy cửa.\par

Đồng bằng sông Cửu Long gồm thành phố Cần Thơ (được coi là thủ phủ miền Tây) và 12 tỉnh Long An, Tiền Giang, Bến Tre, Vĩnh Long, Trà Vinh, Hậu Giang, Sóc Trăng, Đồng Tháp, An Giang, Kiên Giang, Bạc Liêu và Cà Mau với tổng diện tích 41,000 $km^2$ và tổng dân số là 20 triệu dân. Theo thống kê từ Cục trồng trọt (tháng 03/2019) tổng diện tích trồng cây ăn trái ở các tỉnh phía Nam đạt hơn 569,000 ha (chiếm 60\% diện tích cả nước), sản lượng ước đạt 6,6 triệu tấn (chiếm 67\% sản lượng cả nước), thống kê 14 loại cây ăn trái được trồng hơn 10,000 ha thì trong đó cam, bưởi đạt diện tích 44,000 ha. Đặc biệt, cam là một trong những loại quả được sử dụng nhiều nhất trên Thế giới được trồng nhiều  ở các tỉnh Đồng Tháp, Hậu Giang và thành phố Cần Thơ, vì nó giàu chất chống oxy hóa và chất phytocchemical. Theo các nhà khoa học Anh: ``Bình quân trong một trái cam có chứa khoảng 170 mg phytocchemicals bao gồm các chất dưỡng da và chống lão hóa''. Giá trị dinh dưỡng trong mỗi quả cam gồm: Mỗi 100 gr quả cam có chứa 87,6 g nước, 1.104 microgram Carotene -- một loại vitamin chống oxy hóa, 30 mg vitamin C, 10,9 g chất tinh bột, 93 mg kali, 26 mg canxi, 9 mg magnesium, 0,3 g chất xơ, 4,5 mg natri, 7 mg Chromium, 20 mg phốt pho, 0,32 mg sắt và giá trị năng lượng là 48 kcal \cite{bvdk-hongngoc}. 

Theo đó, tình hình dịch bệnh trên cây ăn trái xuất hiện, gây thiệt hại nặng nề cho các chủ nhà vườn. Cụ thể tại tỉnh Đồng Tháp, theo trang Đồng Tháp Online (07/2018) ``Theo thống kê của Phòng Nông nghiệp và Phát triển nông thôn huyện Lai Vung, dịch bệnh vàng lá thối rễ xuất hiện trên cây có múi từ năm 2016 đến nay, đã gây thiệt hại 260 ha diện tích vườn cam, quýt tỷ lệ thiệt hại 50 -- 80\%'' \cite{dongthaponline}. Cùng thời điểm, trang Vĩnh Long Online đưa tin ``Ở các tỉnh Vĩnh Long, Trà Vinh, Hậu Giang, v.v. số lượng vườn cây có múi bị nhiễm bệnh khá lớn. Theo các ngành chức năng tỉnh Hậu Giang, toàn tỉnh có hàng ngàn héc-ta vườn cây có múi bị bệnh vàng lá (chủ yếu là cam sành); loại bệnh này giống như bị ung thư nên dù nông dân chữa trị nhiều cách vẫn không khỏi được'' \cite{vinhlongonline}.\par

Theo Thông báo tình hình sinh vật gây hại chủ yếu 7 ngày của Cục bảo vệ thực vật (từ ngày 09-15/03/2019) ở cây ăn quả có múi, ``Bệnh Greening (vàng lá gân xanh) diện tích nhiễm bệnh 2.055 ha (tăng 105 ha so với kỳ trước, giảm 642 ha so với cùng kỳ năm trước), nặng 87 ha. Tập trung chủ yếu tại các tỉnh Tiền Giang, Hậu Giang, Vĩnh Long, Kiên Giang, Bình Phước, Nghệ An.'' cho thấy tình hình sinh vật gây hại đang diễn biến phức tạp và đa dạng gây khó khăn trong quá trình nhận dạng bằng mắt thường dẫn đến tình trạng nhận dạng nhầm hoặc khó nhận dạng ở thời gian đầu phát bệnh cho các chủ nhà vườn.\par 

Vì vậy, tìm một phương pháp nhanh, tự động, ít tốn kém và  chính xác để phân loại các trường hợp bệnh trên cây ăn quả (cam) có ý nghĩa thực tế rất lớn với nền kinh tế nước nhà.\par


\section{Những nghiên cứu liên quan}
Các nghiên cứu về xử lý ảnh trước đây được dùng như một cơ chế phát hiện bệnh \cite{weizheng2008grading}\cite{ei2004integrating}. Từ cuối những năm 1970, công nghệ xử lý hình ảnh dựa trên máy tính được áp dụng trong nghiên cứu kỹ thuật nông nghiệp trở nên phổ biến \cite{weizheng2008grading}\cite{moshashai2008identification}.

Trong bài báo \cite{al2011detection} trình bày nghiên cứu trước đây trong phát hiện và phân loại bệnh trên lá. Chuyển đổi ảnh từ không gian ảnh màu (RGB) sang không gian màu độc lập với thiết bị (CIELAB). Sử dụng kỹ thuật gom cụm K-means [Diday et al., 1979] để phân đoạn ảnh. Tình toán các đặc trưng kết cấu từ ảnh đã được phân đoạn. Trích các đặc trưng thông qua mạng thần kinh được đào tạo trước (pre-trained neural network). Mô hình mạng thần kinh được dùng để phát hiện và phân loại các bệnh trên lá.\par

Trong luận văn này, chúng tôi đề xuất đề tài ứng dụng máy học véc-tơ hỗ trợ trong phân loại bệnh trên lá cam sử dụng các đặc trưng  SIFT [Lowe, 2004] kết hợp mô hình túi đựng từ trực quan (BoVW), Color [Michael, 1991], HOG [Dalal \& Triggs, 2005], GIST [Oliva \& Torralba, 2001], ResNet [He et al., 2015] và huấn luyện mô hình máy học SVM [Vapnik, 1995]. Thực hiện rút trích các đặc trưng (được đề xuất) từ ảnh bệnh và huấn luyện mô hình máy học SVM để phân loại. Kết quả thực nghiệm trên tập dữ liệu gồm 711 ảnh của 5 loại, trong đó 4 loại lá bệnh và 1 loại lá khỏe lần lượt là bệnh Ghẻ nhám, Rầy phấn trắng, Vàng lá gân xanh, Vàng là thối rễ và Lá khỏe được thu thập tại tỉnh Đồng Tháp và tỉnh Hậu Giang.


\section{Mục tiêu của đề tài}
Mục tiêu của đề tài \emph{Ứng dụng máy học véc-tơ hỗ trợ trong phân loại bệnh trên lá cam} bao gồm ba mục tiêu. 
\begin{itemize}
\item[-] Thứ nhất, xây dựng được một mô hình phân loại. 
\item[-] Thứ hai, độ chính xác chấp nhận được. 
\item[-] Cuối cùng là thời gian phân loại không quá lâu.
\end{itemize}

\section{Đối tượng và phạm vi nghiên cứu}
\subsection{Đối tượng nghiên cứu}
Đối tượng nghiên cứu của đề tài là các hình ảnh lá cam bệnh được chụp từ camera:
\begin{itemize}
	\item[-] Thời gian chụp khi lá cam bắt đầu nhiễm bệnh hoặc đang bị nhiễm bệnh.
	\item[-] Ảnh chụp trực diện lá cam bị bệnh.
\end{itemize}

\subsection{Phạm vi nghiên cứu}
Trong đề tài, chúng tôi tìm hiểu về đặc điểm màu sắc, hình dạng của bốn loại bệnh (Ghẻ nhám, Rầy phấn trắng, Vàng lá gân xanh và Vàng lá thối rễ). Phần kỹ thuật, tìm hiểu các phương pháp trích chọn đặc trưng SIFT kết hợp túi đựng từ trực quan (BoVW), Color, HOG, GIST, ResNet và mô hình máy học SVM.

\section{Phương pháp nghiên cứu}
\subsection{Nghiên cứu lý thuyết}
Chúng tôi nghiên cứu nguyên nhân cơ bản, cũng như hiện tượng một số loại bệnh trên lá cam. Các mô hình màu, mô hình biểu diễn ảnh, phương pháp trích chọn đặc trưng ảnh. Mô hình máy học SVM được chúng tôi nghiên cứu nhằm mục đích xây dựng mô hình phân loại bệnh.

\subsection{Thực nghiệm}
Bắt đầu từ việc thu thập hình ảnh bệnh trên lá cam, sau đó xây dựng các mô đun trích chọn đặc trưng, nối các véc-tơ đặc trưng của từng mô đun thành một véc-tơ có 103124 thành phần và cuối cùng là huấn luyện mô hình phân lớp SVM. Tiếp theo, chúng tôi tiến hành chạy thực nghiệm, so sánh kết quả với mô hình máy học KNN và đánh giá kết quả của hệ thống.


\section{Nội dung nghiên cứu}
Các lý thuyết về các mô hình màu (RGB, HSV); các mô hình biểu diễn ảnh (raster, vector); các kỹ thuật trích đặc trưng (SIFT, Color, HOG, 	GIST và ResNet); mô hình máy học SVM của thư viện scikit-learn [Cournapeau, 2007]. Ngôn ngữ lập trình Python [Rossum, 1991], thư viện mã nguồn mở OpenCV [Bradski \& Kaehler, 2012], thư viện scikit-image \cite{scikit-image}, thư viện Pyleargist [Grisel, 2009] và thư viện Keras [Chollet, 2015].


\section{Bố cục luận văn}
Phần còn lại của luận văn được tổ chức như sau. Chúng tôi trình bày các khái niệm về kỹ thuật được dùng xây dựng hệ thống trong chương 2. Các bước xây dựng một hệ thống nhận dạng và phân loại được trình bày trong chương 3. Kết quả thực nghiệm sẽ được trình bày trong chương 4 trước khi kết luận và hướng phát triển được trình bày trong chương 5.







